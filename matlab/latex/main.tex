
\documentclass[xcolor={dvipsnames}]{beamer}
\usepackage{amsmath,amsfonts,amssymb,pxfonts,eulervm,xspace}
\usepackage{graphicx}
 \usepackage{multimedia}
\usepackage{media9}
\usepackage{minted}

\graphicspath{{./figures/}}
\usetheme{ccnycrest}


\newenvironment{changemargin}[2]{%
\begin{list}{}{%
\setlength{\topsep}{0pt}%
\setlength{\leftmargin}{#1}%
\setlength{\rightmargin}{#2}%
\setlength{\listparindent}{\parindent}%
\setlength{\itemindent}{\parindent}%
\setlength{\parsep}{\parskip}%
}%
\item[]}{\end{list}}

\begin{document}

\title{Matlab \Leftrightarrow Python}

\begin{frame}
	\titlepage
\end{frame}

\begin{frame}[fragile]{}

\begin{block}{Matlab}
\begin{minted}{matlab}
disp("Hello World!");
\end{minted}
\end{block}

\begin{block}{Python}
\begin{minted}{python}
print("Hello World!")
\end{minted}
\end{block}
\end{frame}

\begin{frame}[fragile]{Comments}

\begin{block}{Matlab}
\begin{minted}{matlab}
% this is a comment

%{
   This is a long comment in Matlab 7
%}
\end{minted}
\end{block}

\begin{block}{Python}
\begin{minted}{python}
# this is a comment

"""
   This is a long comment 
"""
\end{minted}
\end{block}
\end{frame}


\begin{frame}[fragile]{Variables}
\begin{block}{Matlab}
\begin{minted}{matlab}
x = 5.71;
I = besseli(x,A);
A = [1 2 3; 4 5 6; 7 8 9];
\end{minted}
\end{block}

\begin{block}{Python}
\begin{minted}{python}
x = 5.71

import numpy as np
A = np.array([[1, 2, 3], [4, 5, 6], [7, 8, 9]])

I = besseli(x,A)
\end{minted}
\end{block}
\end{frame}

\begin{frame}[fragile]{String Formatting}

\begin{block}{Matlab}
\begin{minted}{matlab}
A = pi*ones(1,3);
txt = sprintf('%f | %.2f | %12f', A);
\end{minted}
\end{block}

\begin{block}{Python}
\begin{minted}{python}
A = np.pi*np.ones(3)
text = "{:f} | {:.2f} | {:.12f}".format(*A)
\end{minted}
\end{block}
\end{frame}

\begin{frame}[fragile]{User Input}

\begin{block}{Matlab}
\begin{minted}{matlab}
x = input(prompt);
str_input = input(prompt,'s');
\end{minted}
\end{block}

\begin{block}{Python}
\begin{minted}{python}
str_input = input(prompt)
\end{minted}
\end{block}
\end{frame}

\begin{frame}[fragile]{Selection Statements}

\begin{columns}
\column{0.5\textwidth}
\begin{block}{Matlab}
\begin{minted}{matlab}
if x > 5
    y = 2;
elseif x < 0
    y = 8;
else
    y=24;
end
\end{minted}
\end{block}

\column{0.5\textwidth}
\begin{block}{Python}
\begin{minted}{python}
if x>5:
    y=2
elif x<0:
    y=8
else:
    y=24
\end{minted}
\end{block}
\end{columns}
\end{frame}

\begin{frame}[fragile]{For Loops}

\begin{columns}
\column{0.5\textwidth}
\begin{block}{Matlab}
\begin{minted}{matlab}
x = ones(1,10);
for n = 2:6
    x(n) = 2 * x(n - 1);
end
\end{minted}
\end{block}

\column{0.5\textwidth}
\begin{block}{Python}
\begin{minted}{python}
x = np.ones(10)
for n in range(2,7):
    x[n] = 2*x[n-1]
    
\end{minted}
\end{block}
\end{columns}
\end{frame}

\begin{frame}[fragile]{While Loops}


\begin{block}{Matlab}
\begin{minted}{matlab}
n = 1;
nFactorial = 1;
while nFactorial < 1e100
    n = n + 1;
    nFactorial = nFactorial * n;
end
\end{minted}
\end{block}

\begin{block}{Python}
\begin{minted}{python}
n = 1
nFactorial = 1
while nFactorial < 1e100:
    n+=1
    nFactorial*=n

\end{minted}
\end{block}
\end{frame}

\begin{frame}[fragile]{Defining Functions}

\begin{block}{Matlab}
\begin{minted}{matlab}
function f = fact(n)
    f = prod(1:n);
end
\end{minted}
\end{block}

\begin{block}{Python}
\begin{minted}{python}
def fact(n):
    f = np.prod(np.arange(1,n+1))
    return f
\end{minted}
\end{block}

\end{frame}


\begin{frame}[fragile]{Using Functions}

\begin{block}{Matlab}
\begin{minted}{matlab}
% Put file (usually func_name.m) in same folder or on path 
% Just call the function

myfunction(x);
\end{minted}
\end{block}

\begin{block}{Python}
\begin{minted}{python}
from myfile import myfunction

myfunction(x)
\end{minted}
\end{block}
\end{frame}

\begin{frame}[fragile]{Creating Classes: Matlab}
\begin{minted}{matlab}
classdef BasicClass
   properties
      Value
   end
   methods
        function obj = BasicClass(val)
            if val > 0
                obj.Value = val;
            else
                error('Value must be numeric')
            end
        end
        ...
    end
end
\end{minted}
\end{frame}

\begin{frame}[fragile]{Creating Classes: Matlab cont.}
\begin{minted}{matlab}
classdef BasicClass
        ...
    methods
        ...
        function r = roundOff(obj)
            r = round([obj.Value],2);
        end
        function r = multiplyBy(obj,n)
            r = [obj.Value] * n;
        end
    end
end
\end{minted}
\end{frame}

\begin{frame}[fragile]{Creating Classes: Python}
\begin{minted}{python}
class BasicClass(object):
    def __init__(self, value):
        if val>0:
            self.value = value
        else:
            raise ValueError("Value must be positive")
            
    def roundOff(self):
        return round(self.Value, 2)
    
    def multiplyBy(self, n):
        return self.Value * n
\end{minted}
\end{frame}

\begin{frame}[fragile]{Creating Objects}

\begin{block}{Matlab}
\begin{minted}{matlab}
a = BasicClass(pi/3);
roundOff(a);
multiplyBy(a,3);
a.multiplyB(3);
\end{minted}
\end{block}
\begin{block}{Python}
\begin{minted}{python}
a = BasicClass(np.pi/3)
a.Value = np.pi/2
a.roundOff()
a.multiplyBy(3)
\end{minted}
\end{block}
\end{frame}

\begin{frame}{How do we use this info?}
\begin{description}
\item[Get Python]
\url{https://www.continuum.io/downloads}
\item[Translate]
\url{https://docs.scipy.org/doc/numpy-dev/user/numpy-for-matlab-users.html}
\item[Python in Matlab] \url{https://www.mathworks.com/help/matlab/call-python-libraries.htm}
\item[Matlab in Python]
\url{https://www.mathworks.com/help/matlab/matlab_external/call-matlab-functions-from-python.html}
\end{description}
\end{frame}

\end{document}

